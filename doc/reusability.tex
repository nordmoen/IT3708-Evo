\section{Code Modularity}\label{sec:modularity}
The core code in this project is very modular as I hope was somewhat clear in
the previous section(\ref{sec:description}). To expand on that further I will be
giving some examples here, but also try and explain the main drawbacks making
this somewhat less modular.

Since most of the code is designed around classes most of the job extending this
is already done. For future implementations all that is needed is to create a
subclass, implement the function appropriate for that superclass and that is
mostly it.

The main thing holding back, as I mentioned in section \ref{sec:architecture},
is that the interface needs to be updated to accommodate new classes. This is
mainly because Python does not support importing classes at runtime in a more
nice manner than what is supported currently. This means that to include new
classes one must import them in the main file and then extend the interface to
support this. This is quite cumbersome, but the benefits of a command-line
interface is better than the alternatives, at least for me.

In the listing \ref{code:random-bit-fitness} we can see how the modularity comes
in to play. Since all the other code in the projects expects that a fitness
function can just be called, all this new code has to do is to extend the main
fitness class. Since this new fitness class is dependant upon a target value it
need to define its own constructor, but then all that is needed is to implement
"sub\_eval".

\lstinputlisting[language=Python, firstline=17, lastline=31,
	label=code:random-bit-fitness, caption=An extension of the fitness class
which support checking against a random target]{../fitness.py}
