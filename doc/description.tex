\section{Description}\label{sec:description}
My code in this project is designed around a couple of core classes which make
out the main backbone. These classes are meant to be extended for further use
and as such can not be used by them self. The main work is done in the classes
\textit{select\_mech.py} and \textit{select\_protocol.py}. These two classes deal with
selection mechanisms and selection protocols respectively. The main evolutionary
loop is done in \textit{evolution.py}, but this code is quite standard and very short.
The class \textit{fitness.py} deals with fitness and contains both the main class to be
extended and some classes which I needed to solve the \textit{One-Max problem}.
Both \textit{genome.py} and \textit{phenotypes.py} deals with the low level
representations of genome and phenotypes. For simplicity I have decided that a
phenotype class will contain the genome which created it. The next class of some
importance is \textit{population.py} which contains a number of phenotypes and
some methods for getting statistics from the population. The last file of any
major importance is the \textit{main.py} file which contains the starter motor
for running all of the classes above. Since I have chosen to have a command-line
interface for my code the main file must be changed in order to add new classes.
This is because Python is not really that happy about importing files at
runtime, and the code is somewhat static in that regard. New configurations must
also be added in the main file which is somewhat unfortunate, but the upside is
that the code can be very easily run from the command-line with different
configurations which make it somewhat easy to automate.

\subsection{Selection Mechanisms}\label{sec:selection mechanisms}
Since selection mechanisms are a core part of this project I will here describe
most of the core code which would needed to be extended in order to add new
code.

\lstinputlisting[language=Python, firstline=40,
lastline=51, label=code:select-mech, caption=Main class which selection
mechanisms should inherit from]{../select_mech.py}

If the new selection mechanism does not need any variables it self, it can
contain as little as a new definition of \textit{sub\_sample}. This method
receives an amount variable and an array of phenotypes. The amount variable
signifies how many individuals this selection mechanism should retrieve. The
population is just an array of individuals which this sub\_sample method should
select from.

This class also contains a few handy functions, but they are of little
importance to this report.

\subsection{Selection Protocol}\label{sec:selection protocol}
The selection protocol is the second important class in this project, it is
tasked with creating a new population, by using a selection mechanism to select
parents and children.

\lstinputlisting[language=Python, firstline=8,
lastline=25, label=code:select-protocol, caption=Main class which selection
protocols should inherit from]{../select_protocol.py}

In listing \ref{code:select-protocol} we can see that this shares much of the
same architecture design as in the previous section. This class expects a
selection mechanism class, a float representing the percentage of parents to
select for mating and the percentage of children to be created during mating.
The last two variables only influences variation and so on, as all selection
protocols implemented so far returns the same size population as it is given as
input\footnote{This is done so that population size is the same all through out
a run}.

\lstinputlisting[language=Python, firstline=27,
lastline=40, label=code:select-protocol-2, caption=An example implementation of
a selection protocol]{../select_protocol.py}

In the code snippet \ref{code:select-protocol-2} we can see how a selection
protocol is implemented in this project. First a set of parents are selected
using the selection mechanism chosen as input. Then we create random couples for
mating\footnote{Some other project that I have talked to have used the selection
	mechanism again when creating couples, I have chosen to keep it this way
	to maintain some variation. In some of the selection mechanisms this can
	be controlled, but I think this should not hold this project back that
much. If needed this can easily be changed.}.

\subsection{Architecture}\label{sec:architecture}
The rest of the code follows very much in the same path as the code shown above.
By using inheritance we can get a lot of different implementation, which can
still be used by code not specially design for the new implementations. As can
be seen in the code above I have included a some amount of assertions to try and
control what the subclasses do\footnote{By using some optimization flags for
	Python these will be ignored. This can be handy when wanting to run the
code faster}. Most of the files in this project contains one class which is
designed to be extended, because of the length requirement on this report
I have not included all those classes.

The only code not following this "standard" is the main class which only job is
to parse the command-line input and start the evolutionary loop. Because
selection mechanisms, protocols and all the variables associated with them can
be configured on the command-line this is sort of the weakest link in the
project. Since new implementations of selection mechanisms must be included here
the main file must also be changed. But this is not something which affects the
rest of the code, if I had chosen another user interface style this could be
avoided.
